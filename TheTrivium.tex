%% outline-sample.tex
%% Copyright 1991 Peter Halvorson
%% Updates for LaTeX2e copyright 2002 Seth Flaxman
%% Updated for LPPL 1.3c or later by Clea F. Rees (for Seth Flaxman), 2008/10/06.
%
% This work may be distributed and/or modified under the
% conditions of the LaTeX Project Public License, either version 1.3
% of this license or (at your option) any later version.
% The latest version of this license is in
%   http://www.latex-project.org/lppl.txt
% and version 1.3 or later is part of all distributions of LaTeX
% version 2005/12/01 or later.
%
% This work has the LPPL maintenance status `unmaintained'.
%
% This work consists of the files outline.sty and outline-sample.tex.
% Save file as: outline-sample.tex

\documentclass{report}
\usepackage{outline}
\usepackage[margin=1in]{geometry}

% [outline] includes new outline environment. I. A. 1. a. (1) (a)
% use \begin{outline} \item ... \end{outline}

\pagestyle{empty}

\begin{document}

\begin{outline}
  \item {\bf The Liberal Arts}
  \begin{outline}
    \item {\bf Trivium} - pertain to mind
    \begin{outline}
       \item Logic
       \item Grammer
       \item Rhetoric
     \end{outline}
    \item {\bf Quadrivium} - pertain to matter
    \begin{outline}
       \item Arithmetic
       \item Geometry
       \item Astronomy
       \item Music
     \end{outline}
     \item {\bf Language Arts} - the correct and effective use of language
    \begin{outline}
       \item \textit{Phonetics} - how to combine sounds and form spoken words properly
       \item \textit{Spelling} - how to combine letters and form written words properly
       \item \textit{Grammar} - how to combine words to form sentences properly
       \item \textit{Rhetoric} - how to combine sentences into larger structures
       \item \textit{Logic} - how to combine concepts in a truthful, reasoned way
     \end{outline}
    \item {\bf Norms of Language Arts} 
    \begin{outline}
       \item \textit{Correctness} - the norm of phonetics, spelling, and grammar
       \item \textit{Effectiveness} - the norm of of rhetoric
       \item \textit{Truth} - the norm of logic
     \end{outline}
  \end{outline}
  \item {\bf The Nature and Function of Language }
  \begin{outline}
    \item {\bf Means of Communication }
    \begin{outline}
      \item {\bf Imitation} - an artificial likeness, for example:
      \begin{outline}
        \item Paintings
        \item Photographs
        \item Statues
      \end{outline}
      \item {\bf Symbol} - an arbitrary sign upon which meaning is imposed
      \begin{outline}
	\item \textit{Common} - a symbol of a common language (e.g., English)
	\item \textit{Special} - a symbol of a specialized language (e.g., chemistry)
      \end{outline}
    \end{outline}
    \item {\bf Terms of Essence }
    \begin{outline}
      \item \textit{Essence} - that which makes a being what it is
      \item \textit{Species} - the set of all beings sharing an essence
      \item \textit{Genus} - a set of more than one species
      \item \textit{Aggregate} - a set of more than one individuals
    \end{outline}
    \item {\bf Imposition and Intention}
    \begin{outline}
      \item Imposition
      \begin{outline}
        \item \textit{Zero imposition} - discussing properties of a sign other than its meaning
        \item \textit{First imposition} - using a word only in relation to its meaning
        \item \textit{Second imposition} - refers both to the sign and the meaning (grammar)
      \end{outline}
      \item Intention
      \begin{outline}
        \item \textit{First intention} - refers to reality
        \item \textit{Second intention} - refers to the concept (logic)\\ \\
      \end{outline}
    \end{outline}
  \end{outline}
  \item {\bf General Grammer}
  \begin{outline}
    \item{\bf Categorematic and Syncategorematic Words}
    \begin{outline}
      \item \textit{Categorematic words} - significant by themselves
      \begin{outline}
        \item \textit{Substantives} - nouns, pronouns
        \item \textit{Attributives} - verbs, adjectives (primary); adverbs (secondary)
      \end{outline}
      \item \textit{Syncategorematic words} - only significant in combination with other words
      \begin{outline}
        \item \textit{Definitives} - articles, pronomials
        \item \textit{Connectives} - prepositions, conjunctions
        \item \textit{Copula} - connects subject and predicate
      \end{outline}
    \end{outline}
  \end{outline}
  \item {\bf Terms and Their Grammatical Equivalents}
  \item {\bf Propositions and Their Grammatical Expression}
  \item {\bf Relations of Simple Propositions}
  \item {\bf The Simple Syllogism}
  \item {\bf Relations of Hypothetical and Disjunctive Propositions}
  \item {\bf Fallacies}
  \item {\bf A Brief Summary of Inductions}
  \item {\bf Composition and Reading}
\end{outline}

\end{document}
