%% outline-sample.tex
%% Copyright 1991 Peter Halvorson
%% Updates for LaTeX2e copyright 2002 Seth Flaxman
%% Updated for LPPL 1.3c or later by Clea F. Rees (for Seth Flaxman), 2008/10/06.
%
% This work may be distributed and/or modified under the
% conditions of the LaTeX Project Public License, either version 1.3
% of this license or (at your option) any later version.
% The latest version of this license is in
%   http://www.latex-project.org/lppl.txt
% and version 1.3 or later is part of all distributions of LaTeX
% version 2005/12/01 or later.
%
% This work has the LPPL maintenance status `unmaintained'.
%
% This work consists of the files outline.sty and outline-sample.tex.
% Save file as: outline-sample.tex

\documentclass{report}
\usepackage{outline}
\usepackage[margin=1in]{geometry}

% [outline] includes new outline environment. I. A. 1. a. (1) (a)
% use \begin{outline} \item ... \end{outline}

\pagestyle{headings}

\begin{document}
\title{An Outline of Sister Miriam Joseph's\\\textit{The Trivium}}
\author{Shane Michael Sexton}

\maketitle

\begin{outline}
  \item {\bf The Liberal Arts}
  \begin{outline}
    \item {\bf Trivium} - pertains to mind
    \begin{outline}
       \item Logic
       \item Grammar
       \item Rhetoric
     \end{outline}
    \item {\bf Quadrivium} - pertains to matter
    \begin{outline}
       \item Arithmetic
       \item Geometry
       \item Astronomy
       \item Music
     \end{outline}
     \item {\bf Language Arts} - the correct and effective use of language
    \begin{outline}
       \item \textit{Phonetics} - how to combine sounds and form spoken words properly
       \item \textit{Spelling} - how to combine letters and form written words properly
       \item \textit{Grammar} - how to combine words to form sentences properly
       \item \textit{Rhetoric} - how to combine sentences into larger structures
       \item \textit{Logic} - how to combine concepts in a truthful, reasoned way
     \end{outline}
    \item {\bf Norms of Language Arts} 
    \begin{outline}
       \item \textit{Correctness} - the norm of phonetics, spelling, and grammar
       \item \textit{Effectiveness} - the norm of of rhetoric
       \item \textit{Truth} - the norm of logic
     \end{outline}
  \end{outline}
  \item {\bf The Nature and Function of Language }
  \begin{outline}
    \item {\bf Means of Communication }
    \begin{outline}
      \item {\bf Imitation} - an artificial likeness, for example:
      \begin{outline}
        \item Paintings
        \item Photographs
        \item Statues
      \end{outline}
      \item {\bf Symbol} - an arbitrary sign upon which meaning is imposed
      \begin{outline}
	\item \textit{Common} - a symbol of a common language (e.g., English)
	\item \textit{Special} - a symbol of a specialized language (e.g., chemistry)
      \end{outline}
    \end{outline}
    \item {\bf Terms of Essence }
    \begin{outline}
      \item \textit{Essence} - that which makes a being what it is
      \item \textit{Species} - the set of all beings sharing an essence
      \item \textit{Genus} - a set of more than one species
      \item \textit{Aggregate} - a set of more than one individuals
    \end{outline}
    \item {\bf Imposition and Intention}
    \begin{outline}
      \item Imposition
      \begin{outline}
        \item \textit{Zero imposition} - discussing properties of a sign other than its meaning
        \item \textit{First imposition} - using a word only in relation to its meaning
        \item \textit{Second imposition} - refers both to the sign and the meaning (grammar)
      \end{outline}
      \item Intention
      \begin{outline}
        \item \textit{First intention} - refers to reality
        \item \textit{Second intention} - refers to the concept (logic)
      \end{outline}
      \newpage
    \end{outline}
  \end{outline}
  \item {\bf General Grammer}
  \begin{outline}
    \item{\bf Categorematic vs. Syncategorematic Words}
    \begin{outline}
      \item \textit{Categorematic words} - significant by themselves
      \begin{outline}
        \item \textit{Substantives} - nouns, pronouns
        \item \textit{Attributives} - verbs, adjectives (primary); adverbs (secondary)
      \end{outline}
      \item \textit{Syncategorematic words} - only significant in combination with other words
      \begin{outline}
        \item \textit{Definitives} - articles, pronomials
        \item \textit{Connectives} - prepositions, conjunctions
        \item \textit{Copula} - connects subject and predicate
      \end{outline}
    \end{outline}
    \item {\bf Substantives}
      \begin{outline}
      \item Concrete vs. Abstract
        \begin{outline}
          \item \textit{Concrete} - an existing object (e.g., ``woman")
          \item \textit{Abstract} - a conception (e.g., ``femininity")
        \end{outline}
      \item Characteristics of Substantives
      \begin{outline}
        \item Number (singular or plural)
        \item Gender (masculine, feminine, neuter)
        \item Person
        \begin{outline}
          \item \textit{First person} - speaker
          \item \textit{Second person} - receiver of speech
          \item \textit{Third person} - person spoken of
        \end{outline}
        \item Case
        \begin{outline}
          \item \textit{Nominative} - performs the action
          \item \textit{Genitive} - possessor
          \item \textit{Dative} - the term to which the action proceeds
          \item \textit{Accusative} - receives the action
        \end{outline}
      \end{outline}
    \end{outline}
    \item {\bf Attributives}
      \begin{outline}
        \item Verbs
          \begin{outline}
            \item Express an attribute with a sense of time; make an assertion
            \item \textit{Transitive} verbs flow from subject to object; \textit{intransitive} verbs stay with the agent
            \item \textit{Tense} - temporal relation between the act and its being spoken of
            \item \textit{Mood} - expresses relation between subject and predicate
            \begin{outline}
              \item \textit{Indicative} - expresses a matter of fact
              \item \textit{Potential} - expresses a possibility
              \item \textit{Interrogative} - requests information
              \item \textit{Volitive} - expresses a wish or desire
            \end{outline}
           \end{outline}
        \item Verbals
          \begin{outline}
            \item Do not assert or express mode
            \item \textit{Infinitives} - standard dictionary form of a verb (e.g., ``to take")
            \item \textit{Gerunds} -  end in ``-ing" and function as nouns
            \item \textit{Participles} - end in ``-ed" or ``-ing" and function as adjectives
          \end{outline}
        \item Adjectives - Unlike verbs or verbals, adjectives express attributes with no notion of time
        \item Adverbs - \textit{Secondary} attributives in that they modify primary attributives (e.g., verbs and adjectives)
        \end{outline}   
        \newpage
    \item {\bf Definitives} - single out an individual (``this") or group (``those")
    \item {\bf Connectives} - connect words and sentences
      \begin{outline}
        \item \textit{Prepositions} - connect words and show their relation (e.g., ``in" or ``behind")
        \item \textit{Conjunctions} - join independent clauses or sentences
          \begin{outline}
            \item May {\bf conjoin}, that is, join sentences and meaning (e.g., ``and")
            \item May {\bf disjoin}, or join sentences but not meanings (e.g., ``but" or ``or")
          \end{outline}
        \end{outline}
    \item {\bf The Pure Copula} - links a subject with a predicate (e.g., ``The book {\bf is} on the table.")
  \end{outline}
  \item {\bf Terms and Their Grammatical Equivalents}
    \begin{outline}
      \item Empirical vs. General Terms
      \begin{outline}
      \item \textit{Empirical} terms designate a specific individual or group (e.g., ``This cat is purring")
      \item \textit{General} terms signify something universal or essentials (e.g., ``Cats purr")
      \end{outline}
      \item Positive vs. negative terms (e.g., ``awake" vs. ``unawake," ``blue" vs. ``non-blue")
      \item Concrete vs. abstract terms (e.g., ``human" vs. ``humanness")
      \item Absolute vs. relative terms
        \begin{outline}
          \item \textit{Absolute} terms can be understood on their own (``man", ``woman")
          \item \textit{Relative} terms come as pairs and are understood in relation to each other (``husband",``wife")
          \end{outline}
        \item Collective vs. distributive terms
          \begin{outline}
            \item \textit{Collective} terms apply to a group as a single entity (``The platoon marched its way north.")
            \item \textit{Distributive} terms apply to a group's individual members (``The platoon marched their way north.")
            \end{outline}
            \item Extension and Intension
            \begin{outline}
              \item \textit{Extension} refers to the complete set of objects to which a term applies (prime numbers are 2, 3, 5, 7, 11...)
              \item \textit{Intension} refers to the essential meaning (prime numbers are numbers greater than 1 whose factors are only 1 and themselves)
              \end{outline}
    \end{outline}
  \item {\bf Propositions and Their Grammatical Expression}
  \item {\bf Relations of Simple Propositions}
  \item {\bf The Simple Syllogism}
  \item {\bf Relations of Hypothetical and Disjunctive Propositions}
  \item {\bf Fallacies}
  \item {\bf A Brief Summary of Inductions}
  \item {\bf Composition and Reading}
\end{outline}

\end{document}
